\section{Concepts, ADT \& Algorithms}
\subsection{Evaluationg Postfix Expressions}
Expressions can also be represented using postfix notations - where an operator comes after its operands
- The advantage of postfix notation is that the order of operation evaluation is unique without the need for precendence rules or parentheses
Example:
% \begin{table}
% \centering
% \begin{tabular}{|c|c|c|c|}
% \hline
% \textbf{Infix Expression} & \textbf{Postfix Expression}
% \end{table}
\subsection{Indix to Postfix}
Algorithm:
\begin{itemize}
    \item Read a token
    \item If operand, output (enqueue) to queue
    \item If '$()$', push the '$()$' to stack
    \item If operator
    \begin{itemize}
        \item If the operator ar the top has greater than or equal precedence, pop operators from the stack and enqueue to the queue until encounter a '$()$' or the stack is empty
        \item push the new operator
    \end{itemize}
    \item If '$()$', pop and outoput until '$()$' has been popped
    \item Repeat until end of input
    \item Pop rest of stack
\end{itemize}
\textbf{Example}: $7 + 8 \times [(3 - 1) / 2 + 1 \times 2] - 3 \times 4 / (3 - 1)$\\
\textbf{Answer}: $7\quad8\quad3\quad1\quad-\quad2\quad/\quad1\quad2\quad\times\quad+\quad\times\quad+\quad3\quad4\quad\times\quad3\quad\quad1\quad-\quad/\quad-$














\newpage
\section{Algorithmal Complexity Analysis}
\section{Sorting Algorithms}
\section{Searching Algorithms}
\section{Data Structures: Stack, Queue, Pointer, Linked List, Tree}
\section{Data Compression Algorithms}
\section{Hasing Algorithms}
\section{String-matching Algorithms}






































\newpage
\section{Section}

\subsection{Một số lưu ý}

\subsubsection{Cài đặt offline}
Template này yêu cầu cài đặt một số gói (package) nâng cao cho TexStudio:
\begin{itemize}
\item Để gõ thuật toán: \texttt{algorithm} và \texttt{algpseudocode}
\item Để nhúng (chèn) code: \texttt{listings}
\end{itemize}
Các gói này được cài đặt thông qua lệnh
\begin{lstlisting}[language=sh]
sudo apt-get install texlive-full
\end{lstlisting}
Tuy nhiên kích thước gói đâu đó vào khoảng 5GB (!). Vì vậy tốt nhất nên xài Overleaf.

\subsubsection{Sử dụng font khác}
Tham khảo font typefaces tại \href{https://www.overleaf.com/learn/latex/Font_typefaces}{link này}.

\subsubsection{Đánh số chỉ mục bằng chữ số La Mã}
Mở file \texttt{main.tex} và bỏ comment dòng 
\begin{lstlisting}[language=tex]
% \renewcommand{\thesection}{\Roman{section}}
% \renewcommand{\thesubsection}{\thesection.\Roman{subsection}}  
\end{lstlisting}

\subsection{Ví dụ}
Ngày xửa ngày xưa, ở vương quốc VNUHCM - US, có một chàng hoàng tử ngồi cắm đầu viết doc\footnote{Đây là footnote, chú thích lại những gì cần chú ý.}.\\
Mặc định muốn xuống dòng chỉ cần dùng $\backslash\backslash$  (2 lần dấu xẹt huyền).\\
Nếu bạn muốn thụt đầu dòng khi bắt đầu paragraph mới, vào \texttt{main.tex} và disble dòng
\begin{lstlisting}[language=tex]
\setlength{\parindent}{0pt}
\end{lstlisting}

\subsection{First subsection}
\subsubsection{First sub-subsection}
Subsection để ví dụ thôi. Thêm vài ví dụ:
\begin{itemize}
    \item Dùng itemize
    \item Vẫn là itemize
\end{itemize}
Sau đó xài enumerate:
\begin{enumerate}
    \item Dùng enumerate
    \item Vẫn là enumerate
\end{enumerate}
Nhỏ hơn subsubsection thì xài \texttt{paragraph}:

\paragraph{Đây là ví dụ cho paragraph}
Lưu ý là paragraph không nằm trong Mục lục.

\subsection{Chia nhỏ nội dung}
Bạn có thể chia nhỏ nội dung của báo cáo thành các file \texttt{.tex} và dùng lệnh \texttt{input} để chèn vào báo cáo chính. Ví dụ có trong file \texttt{main.tex}.